\chapter*{Introduction}

Pour l'heure, les technologies liées à l'intelligence artificielle et à l'apprentissage automatique ne sont que peu connues du grand public, contrairement à des outils tels que la messagerie ou la navigation chiffrées par exemple.

On peut distinguer au moins trois approches à l'apprentissage automatique et à ses utilisations : du point de vue du chercheur, c'est une théorie solide et complexe, fondée sur des mathématiques de haut niveau; du point de vue de l'étudiant, il s'agit de statistiques couplées à de l'algorithmique; pour le profane, cela ressemble très fortement à de la magie. 

Comment peut-on envisager qu'une machine puisse reconnaître des visages, ou mieux encore, en générer, alors que personne ne lui a explicitement inculqué le concept de visage ? Est-il possible qu'un jour nous n'ayons plus d'emploi à cause de l'intelligence artificielle ?

Nous adopterons ici le point de vue de l'étudiant [FINIR]

Le chapitre "Panorama de l'apprentissage artificiel" présente les notions indispensables de la théorie l'apprentissage. Par la suite, nous nous concentrerons sur les réseaux de neurones profonds dans les chapitres "Entraînement et évaluation d'un réseau de neurones" ainsi que "Grandes architectures des réseaux de neurones". Dans ces premiers chapitres, nous resterons très théoriques, tout en présentant des exemples illustratifs simples. A partir du chapitre "Classification de textes" et "Génération de texte", nous présenterons une implémentation possible d'un classifieur et d'un générateur, tous deux fonctionnant grâce à des réseaux de neurones profonds. On s'appuiera pour cela sur un corpus de textes français classiques.