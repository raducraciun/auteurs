\chapter*{Introduction}

Pour l'heure, les technologies liées à l'intelligence artificielle et à l'apprentissage automatique ne sont que peu connues du grand public, contrairement à des outils tels que la messagerie ou la navigation chiffrées par exemple.

On peut distinguer au moins trois approches à l'apprentissage automatique et à ses utilisations : du point de vue du chercheur, c'est une théorie solide et complexe, fondée sur des mathématiques de haut niveau; du point de vue de l'étudiant, il s'agit surtout de statistiques couplées à de l'algorithmique; pour le profane, c'est simplement de la magie. En fait, on pourrait dire que l'aspect philosophique des problématiques liées à l'intelligence artificielle a été mieux compris que l'aspect technique. Certaines attentes ou craintes sont tout à fait fondées, d'autres, beaucoup moins. Une chose est sûre, l'allure à laquelle progressent les technologies de l'apprentissage rendent caduques l'immense majorité des prédictions, même parmi les experts.

Comment peut-on envisager qu'une machine puisse reconnaître des visages, ou mieux encore, en générer, alors que personne ne lui a explicitement inculqué le concept de visage ? Que veut dire créer de l'art ou imiter un artiste ? Et, sans chercher l'alarmisme, est-il possible qu'un jour nous n'ayons plus d'emploi à cause de l'intelligence artificielle ? Voilà le type de questions qu'on ne peut s'empêcher de se poser, que l'on soit novice ou aguerri en la matière.

Nous adopterons ici le point de vue de l'étudiant et nous intéresserons à l'apprentissage artificiel en général, sans expliciter tous les détails, tout en essayant de mettre en pratique certaines notions. Plus précisément, nous avons souhaité découvrir l'apprentissage profond ainsi que certaines de ses applications au traitement de textes.

Le chapitre "Panorama de l'apprentissage artificiel" présente les notions indispensables de la théorie de l'apprentissage. Par la suite, nous nous concentrerons sur les réseaux de neurones profonds dans le chapitre "Entraînement et évaluation d'un réseau de neurones". Dans ces premiers chapitres, nous resterons très théoriques, tout en présentant des exemples simples. Dans le chapitre "Traitement du langage naturel", nous présenterons une implémentation possible d'un classifieur et d'un générateur, tous deux fonctionnant grâce à des réseaux de neurones profonds. On s'appuiera pour cela sur un corpus de textes français classiques.