\chapter{Grandes architectures de réseaux de neurones}

Les techniques d'apprentissage profond se perfectionnent quasiment quotidiennement, et il est très difficile de faire un état de l'art complet à un instant donné. Toutefois, les architectures présentées ici sont classiques et souvent utilisées, seules ou combinées à d'autres. Les réseaux convolutifs sont souvent utilisés dans la classification d'images. Quant aux réseaux récurrents, leurs utilisation est très orientée sur les séries temporelles ou les chaînes de caractères, notamment pour le traitement du langage naturel.

\section{Réseaux à propagation avant}

\section{Réseaux convolutifs}

\section{Réseaux récurrents}

\section{Conclusion}
